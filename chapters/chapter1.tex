\documentclass[../main.tex]{subfiles}

\begin{document}
	
\chapter{CRM architectural differences between constitutive and variable genes}
\label{chapter1}

\section{Introduction}
\label{chapter1:introduction}
There are many known architectural differences between constitutive and variable genes.
In mammals many differences have been observed. CDSs evolve more slowly in constitutive genes than variable genes \autocite{zhangMammalianHousekeepingGenes2004}.
Constitutive promoters are enriched for CpG islands over variable promoters \autocite{carninciGenomewideAnalysisMammalian2006}.
There is conflicting evidence for some differences.
For example, it has been reported in pigs that constitutive promoters are more conserved than variable promoters \autocite{weiCharacterizationGenePromoters2019} whereas in human variable promoters were found to be more conserved than constitutive promoters \autocite{farreHousekeepingGenesTend2007}.
In Arabidopsis, constitutive genes are known to have a broad transcription start region (TSR) while variable genes have a narrow TSR \autocite{mortonPairedEndAnalysisTranscription2014}.
Constitutive genes are known to have a weaker expression in general than variable genes \autocite{czechowskiGenomeWideIdentificationTesting2005,mortonPairedEndAnalysisTranscription2014}.
Constitutive genes are associated with gene body methylation \autocite{zhangGenomewideHighResolutionMapping2006, takunoBodyMethylatedGenesArabidopsis2012,aceitunoRulesGeneExpression2008} while variable gene sare associated with promoter methylation \autocite{zhangGenomewideHighResolutionMapping2006}.
Constitutive promoters are enriched for GA regions while variable promoters are depleted in GA regions \autocite{yamamotoHeterogeneityArabidopsisCore2009}. 


\subsection{GC content}
\label{chapter1:introduction:gc-content}
Mammalian constitutive promoters have a higher GC content than variable promoters \autocite{vinogradovDNAHelixImportance2017,weiCharacterizationGenePromoters2019}.
Hypothesis: Arabidopsis constitutive promoters will have a higher GC content than variable promoters.
Additionally, TATA-containing promoters contain a larger GC-skew peaking at the TSS than TATA-less promoters \autocite{zuoIdentificationTATATATAless2011}.
In GC-rich regions DNA shows higher bendability \autocite{vinogradovBendableGenesWarmblooded2001,vinogradovDNAHelixImportance2003} and lower nucleosome formation potential \autocite{vinogradovNoncodingDNAIsochores2005} and increased ability to transition from B to Z-DNA conformation \autocite{vinogradovDNAHelixImportance2003} linked to higher expression.  GC-rich sequence has a higher DNase-I sensitivity with mmore open chromatin \autocite{difilippoMappingDNaseIHypersensitive2008}.
AT-rich regions show the opposite.
Additionally, GC-rich regions show an increased mutation rate than AT-rich regions \autocite{vinogradovDNAHelixImportance2017}.
\subsection{TFBS coverage}
\label{chapter1:introduction:tfbs-coverage}
There is conflicting evidence in humans for this.
Constitutive genes were found to have higher DNA entropy (less order) than variable genes, suggesting that variable genes are more complex and have a higher density of CREs than constitutive ones. \autocite{thomasDNAEntropyReveals2015}. 
Constitutive promoters had more bps covered by TFBSs than variable promoters \autocite{mattioliHighthroughputFunctionalAnalysis2019}, suggesting that constitutive genes are more complex.

In Arabidopsis the hypothesis is that TFBS percentage coverage of promoters will be higher in constitutive than variable genes.

\subsection{TF diversity}
\label{chapter1:introduction:tf-diversity}
Conflicting evidence in humans:
Constitutive CRMs will recruit more TFs than responsive CRMs, as they will have more TFBSs each attracting multiple TFs \autocite{mattioliHighthroughputFunctionalAnalysis2019}.
In pigs, more types of motifs were found in variable promoters than in constitutive \autocite{weiCharacterizationGenePromoters2019}.
Hypothesis: Constitutive promoters will have more diverse TFBS than variable.

\subsection{TATA boxes}
\label{chapter1:introduction:tata-boxes}
In animals, TATA boxes are enriched in variable promoters \autocite{engstromGenomicRegulatoryBlocks2007,carninciGenomewideAnalysisMammalian2006}.
Hypothesis: In Arabidopsis, TATA boxes will be enriched in variable genes compared to constitutive genes.
Genes with TATA boxes tend to have shorter 5'UTRs \autocite{molinaGenomeWideAnalysis2005,lichtenbergWordLandscapeNoncoding2009}.
TATA-less promoters are enriched for bidirectionality, suggesting that constitutive promoters are enriched for bidirectionality over variable promoters \autocite{dhadiGenomewideComparativeAnalysis2009}.

\section{Methods}
\label{chapter1:methods}


All data analysis and plotting was done using Python 3~\autocite{pythoncoreteamPythonDynamicOpen2020}.
The Shapiro\hyp{}Wilk normality test~\autocite{shapiroAnalysisVarianceTest1965} and Levene's homogeneity of variance~\autocite{leveneRobustTestsEquality1960} were used to test assumptions for parametric tests.
For non\hyp{}parametric analyses the Kruskal\hyp{}Wallis \textit{H} test \autocite{kruskalUseRanksOneCriterion1952} was used to test for differences between constitutive, variable and control promoters.
If necessary, Dunn's post\hyp{}hoc tests \autocite{dunnMultipleComparisonsUsing1964} were used with Bonferroni adjustment for multiple comparisons.

\subsection{Extraction of \textit{cis}-regulatory modules}\label{chapter1:methods:extraction-of-cis-regulatory-modules}

Promoters were extracted from the Arabidopsis TAIR 10 \autocite{lameschArabidopsisInformationResource2012} genome assembly and Ensembl Plants \autocite{howeEnsemblGenomes20202020} annotation (\href{ftp://ftp.ensemblgenomes.org/pub/release-47/plants/gff3/arabidopsis_thaliana/}{.gff3 release 47 date 08/03/2020}) using a custom Python script (\href{https://github.com/Switham1/PromoterArchitecture/blob/master/src/data_sorting/extract_promoter.py}{\texttt{extract\_promoter.py}}, available at \url{https://github.com/Switham1/PromoterArchitecture}).
Only promoters from genes on Arabidopsis chromosomes 1-5 were extracted.
Promoters were extracted from protein coding genes that did not overlap other protein coding genes using pybedtools \autocite{dalePybedtoolsFlexiblePython2011}.
Promoters were extracted 1000 base pairs (bp) upstream of the longest annotated transcript TSS or until the nearest annotated protein coding gene, and 5'UTRs were included downstream of the TSS up until the closest annotated coding sequence (CDS) ATG start codon.
Genes where the whole promoter overlapped a protein coding gene leaving only part of the 5'UTR non-overlapping were flagged and filtered out.
%This is because coding sequences have different conservation patterns to non-coding regions.%see intro XXX%
Promoters with an upstream gene oriented in the reverse direction less than 2000 bp away were flagged to mitigate for potentially overlapping promoters.
This was because with overlapping promoters it is difficult to determine whether CREs belong to one promoter over the other or are used by both promoters.

\subsection{Transcription factor binding site identification}
\label{chapter1:methods:transcription-factor-binding-site-identification}

The resulting promoter annotations were transformed to bed format using BEDOPS gff2bed~\autocite{nephBEDOPSHighperformanceGenomic2012}, and BEDTools getfasta~\autocite{quinlanBEDToolsFlexibleSuite2010} was used to extract promoter sequences from the reference genome.
Promoters were scanned for DAP\hyp{}seq TFBS motifs~\autocite{omalleyCistromeEpicistromeFeatures2016} using FIMO~\autocite{grantFIMOScanningOccurrences2011} with a zero\hyp{}order background model created using fasta\hyp{}get\hyp{}markov~\autocite{baileyMEMESuiteTools2009}.
A \textit{p}\hyp{}value threshold of \texttildelow0.0001 and max stored sequences \texttildelow5000000 was used, and the output was filtered using a \textit{q}\hyp{}value threshold of 0.05.
Arabidopsis gene IDs for promoters and the TFs binding them were recovered for further analysis.

\subsection{Gene selection}\label{chapter1:methods:gene-selection}

To investigate the stability of expression, \textcite*{czechowskiGenomeWideIdentificationTesting2005} analysed gene expression data from \textit{Arabidopsis thaliana} Col-0 across 79 different tissues, organs, developmental stages and genotypes.
This data enables genes to be ranked according to stability of expression across tissues using coefficient of variation (CV) values.
Only genes which had at least one TFBS found in their promoters using FIMO (see \autoref{chapter1:methods:transcription-factor-binding-site-identification}) were ranked according to CV.
Recreating the methodology used by \textcite*{czechowskiGenomeWideIdentificationTesting2005}, CV was used to select the 100 most constitutively expressed genes from raw expression data generated in their study.
Alongside this, the 100 most variable genes were chosen.
100 promoters were selected from the central distribution of expression CV in the \textcite*{czechowskiGenomeWideIdentificationTesting2005} dataset.
To ensure even coverage, 10 genes were selected randomly from 10 bins covering the range of CV between the constitutive and variable gene sets.


\subsection{Sliding window creation}
\label{chapter1:methods:sliding-window-creation}
%describe how sliding windows were created%
Promoters were split into 100 bp sliding windows with a 50 bp step size using a custom Python script (\href{https://github.com/Switham1/PromoterArchitecture/blob/master/src/rolling_window/rolling_window.py}{rolling\_window.py}, available at \url{https://github.com/Switham1/PromoterArchitecture}).
Windows with fewer than 100 promoters extending to that location were removed.

\subsection{GC content}
{\label{chapter1:methods:gc-content}}

Percentage GC content of promoters and each promoter window was determined using python to test the hypothesis that constitutive genes have a higher GC content than variable genes.
%The Mann\hyp{}Whitney U test~\autocite{mannTestWhetherOne1947}


\subsection{Transcription factor binding site coverage}
{\label{chapter1:methods:transcription-factor-binding-site-coverage}}

To test the hypothesis that the CRMs of variable genes will have a lower percentage of base pairs covered by at least one TFBS than CRMs of constitutive genes, BedTools coverage tool was utilised.
The number of base pairs covered by at least one motif in a given sequence was calculated using the BEDTools coverage tool~\autocite{quinlanBEDToolsFlexibleSuite2010}.
The proportion of base pairs covered by TFBSs in constitutive promoters was compared to variable promoters using a Mann Whitney U test~\autocite{mannTestWhetherOne1947}.

\subsection{Open chromatin coverage}
{\label{chapter1:methods:open-chromatin-coverage}}

Negative control (treated with NaOH) ATAC\hyp{}seq data was downloaded from \textcite{potterCytokininModulatesContextdependent2018} for root and shoot tissues.
Individual bed files for each replicated were concatenated and BedTools merge was used to combine overlapping peaks.
An intersection for root and shoot open chromatin was created using BedTools intersect.
To test the hypothesis that the CRMs of variable genes will have a lower proportion of open chromatin, BedTools coverage tool was used.
The number of base pairs covered by root, shoot or the root\hyp{}shoot intersect open chromatin was calculated using BedTools coverage.


\subsection{TF diversity}
{\label{chapter1:methods:tf-diversity}}

The unique TF count for each promoter and promoter window was calculated \ie{} if TFBSs for a TF were found several times in a promoter, that TF was only counted once.
TFs were only classed as present in a promoter window if the centre of the TFBS was inside the window.
To test the hypothesis that constitutive CRMs will have a more diverse TFBS profile than variable CRMs the Shannon diversity was calculated.
The mapped motif annotations were analysed using the the skbio.diversity.alpha.shannon Python module (\url{https://github.com/biocore/scikit-bio}) to calculate the Shannon diversity of individual TFs and also TF families binding each promoter or promoter window.
The Shannon diversity, unique TFBS counts and raw TFBS counts were analysed comparing constitutively expressed promoters to variable promoters.
The Mann\hyp{}Whitney U test was used~\autocite{mannTestWhetherOne1947}.

As documented in \href{https://github.com/Switham1/PromoterArchitecture/blob/master/src/plotting/TF_diversity_plots_wholeprom.ipynb}{\texttt{TF\_diversity\_plots\_wholeprom.ipynb}}, a table was created containing each promoter on a different row with each TF family in a different column.
The numbers in each cell represent the number of times TFs belonging to a particular TF family are predicted to bind to a certain promoter.
A principle component analysis was run where \SI{95}{\percent} of the variation was maintained with 22 components.
Then hierarchical clustering was used (Python code from \url{http://www.nxn.se/valent/extract-cluster-elements-by-color-in-python}) to estimate the number of clusters, K, to be used in Kmeans clustering.
The number of clusters was predicted using the silhouette method~\autocite{rousseeuwSilhouettesGraphicalAid1987} and then used as K in Kmeans clustering using the sklearn.cluster.KMeans Python module (\url{https://github.com/scikit-learn/scikit-learn/tree/master/sklearn/cluster}).

\subsection{TATA box enrichment}
\label{chapter1:methods:tata-box-enrichment}

TATA box presence/absence for the genes of interest was downloaded from Eukaryotic promoter database (release: At\_EPDnew\_004)~\autocite{dreosEukaryoticPromoterDatabase2017}.
Genomic Association Tester (GAT)~\autocite{hegerGATSimulationFramework2013} was used to
compare enrichment of TATA boxes in constitutive and responsive genes to test the hypothesis that variable genes are enriched for TATA boxes over constitutive genes.
The 15 bp TATA boxes were used as segments of interest. Both the 100 constitutive and 100 responsive promoter annotations were separately tested for enrichment of TATA boxes compared to the background workspace file containing all 200 promoters of interest.

\section{Results}
\label{chapter1:results}

3299 genes were flagged and removed from the analysis as they were overlapping other genes.
An additional 484 genes were filtered that contained no TFBSs in their CRMs when scanned with FIMO. 
An analysis pipeline was created to analyse promoter architecture of constitutive and variable genes.
1959 genes were flagged as having potentially overlapping promoters where the upstream gene was positioned in the opposite direction and was less than 2000 bp away from the TSS.
37 constitutive, 13 variable and 23 control genes were flagged as having potentially overlapping promoters.
The top 100 constitutive and top 100 variable genes chosen as in \textcite{czechowskiGenomeWideIdentificationTesting2005} are annotated on the  coefficient of expression variation distribution of all Arabidopsis promoters after the filtering specified above \autoref{fig:cv-dist-allgenes}.

\begin{figure}[hbt!]
	\begin{center}
		\capstart
		\includegraphics[width=0.70\columnwidth]{genes/czechowski_co-oefficient_of_variation_distribution}
		\caption{
			Distribution of coefficient of expression variation (CV) values for all Arabidopsis thaliana genes after filtering overlapping genes and genes with no TFBSs in their promoters.
			CV values calculated as in \textcite{czechowskiGenomeWideIdentificationTesting2005}.
			The CV range of the top 100 constitutive genes with the lowest CVs are annotated with an arrow (blue range marker).
			The CV range of the top 100 variable genes with the highest CVs are annotated with an orange range marker).			
			\label{fig:cv-dist-allgenes}
		}
	\end{center}
\end{figure}

\subsection{GC content}

The hypothesis that constitutive genes have a higher GC content than variable genes was tested.
GC content was not significantly different between variable, constitutive and control promoters (Kruskal\hyp{}Wallis~\textit{H} = 5.8,~\textit{P} \textgreater{} 0.05) \autoref{fig:gc-content-wholeprom}).

\begin{figure}[hbt!]
	\begin{center}
		\capstart
		\includegraphics[width=0.60\columnwidth]{GC_content/wholeprom/Czechowski_GC_content_box}
		\caption{
			Percentage GC content of 100 constitutive (blue), 100 variable (orange) and 100 control (green) Arabidopsis \textit{cis}\hyp{}regulatory modules (CRMs).
			Promoters were extracted 1000 bp upstream of the annotated Araport 11 \autocite{chengAraport11CompleteReannotation2017} TSS or until the nearest gene.
			5UTRs were extended downstream of the TSS to the closest coding region.  Box plots have box boundaries that represent 25th, 50th (median) and 75th percentiles; whiskers are drawn up to the largest or smallest observed point that falls within 1.5 times the interquartile range.
			\label{fig:gc-content-wholeprom}
		}
	\end{center}
\end{figure}

A sliding window analysis revealed that within the first 400 bp upstream of the ATG start codon the median open chromatin for consitututive genes was \textasciitilde{}\SI{60}{\percent} (\autoref{fig:gc-content-sliding-window}A).
\begin{figure}[hbt!]
	\begin{center}
		\capstart
		\includegraphics[width=1\columnwidth]{GC_content/Czechowski_genetypenocontrol_percentage_GC_content_median_sliding_window}
		\caption{
			Sliding window analysis of 100 constitutive (blue) and 100 variable (orange) Arabidopsis \textit{cis}\hyp{}regulatory modules (CRMs).
			Promoters were extracted 1000 base pairs (bp) upstream of the annotated Araport 11 \autocite{chengAraport11CompleteReannotation2017} TSS or until the nearest gene.
			5UTRs were extended downstream of the TSS to the closest coding region.
			Data points are positioned in the centre of each 100 bp window.
			Windows are offset by 50 bp.
			Shading represents 95 confidence intervals estimated using 10000 bootstraps.
			A: Median percentage of open chromatin peaks overlapping 100 bp windows. Open chromatin peaks derived from the intersect of root and shoot peaks derived from negative control (treated with NaOH) ATAC\hyp{}seq data by \textcite{potterCytokininModulatesContextdependent2018}.	
			B: Median percentage GC content in 100 bp windows. N = 95
			\label{fig:gc-content-sliding-window}
		}
	\end{center}
\end{figure}
%This 400 bp region contained the majority of TSSs (\autoref{fig:all-combined-sliding-window}B).
Within this 400 bp region percentage GC content looked higher for constitutive genes than variable genes (\autoref{fig:gc-content-sliding-window}B).
Further analysis showed that there was a significant difference between promoter types in theis 400 bp region (Kruskal\hyp{}Wallis~\textit{H} = 12.6,~\textit{P} \textless{} 0.01).
Dunn's posthoc tests revealed that within the 400 bp region constitutive promoters had a significantly higher GC content (\SI{35.1}{\percent} ± 5.0) than variable promoters (\SI{33.1}{\percent} ± 4.7; ~\textit{P} \textless{} 0.01) (\autoref{fig:gc-content-400bpprom}.

\begin{figure}[hbt!]
	\begin{center}
		\capstart
		\includegraphics[width=0.60\columnwidth]{GC_content/400bpprom/Czechowski_GC_content_box}
		\caption{
			Percentage GC content of 100 constitutive (blue), 100 variable (orange) and 100 control (green) Arabidopsis \textit{cis}\hyp{}regulatory modules (CRMs) in a 400 bp region upstream of the ATG start codon.
			Box plots have box boundaries that represent 25th, 50th (median) and 75th percentiles; whiskers are drawn up to the largest or smallest observed point that falls within 1.5 times the interquartile range.
			\label{fig:gc-content-400bpprom}
		}
	\end{center}
\end{figure}




\subsection{Transcription factor binding site coverage}
The hypothesis that constitutive genes have a higher TFBS coverage than variable genes was tested.
TFBS coverage was not significantly different between promoter types (Kruskal\hyp{}Wallis~\textit{H} = 5.1,~\textit{P} \textgreater{} 0.05) \autoref{fig:tfbs-coverage-wholeprom}.

\begin{figure}[hbt!]
	\begin{center}
		\capstart
		\includegraphics[width=0.60\columnwidth]{bp_covered/wholeprom/Czechowski_TFBS_coverage_box}
		\caption{
			Percentage TFBS coverage of 100 constitutive (blue), 100 variable (orange) and 100 control (green) Arabidopsis \textit{cis}\hyp{}regulatory modules (CRMs).
			Promoters were extracted 1000 bp upstream of the annotated Araport 11 \autocite{chengAraport11CompleteReannotation2017} TSS or until the nearest gene.
			5UTRs were extended downstream of the TSS to the closest coding region.  Box plots have box boundaries that represent 25th, 50th (median) and 75th percentiles; whiskers are drawn up to the largest or smallest observed point that falls within 1.5 times the interquartile range.		
			\label{fig:tfbs-coverage-wholeprom}
		}
	\end{center}
\end{figure}

A sliding window analysis revealed that in the constitutive open chromatin region 400 bp upstream of the ATG start codon (\autoref{fig:tfbs-coverage-sliding-window}A), TFBS coverage was higher in variable promoters than constitutive (\autoref{fig:tfbs-coverage-sliding-window}B).

 \begin{figure}[hbt!]
	\begin{center}
		\capstart
		\includegraphics[width=1\columnwidth]{bp_covered/Czechowski_genetypenocontrol_percentage_bases_covered_median_sliding_window}
		\caption{
			Sliding window analysis of 100 constitutive (blue) and 100 variable (orange) Arabidopsis \textit{cis}\hyp{}regulatory modules (CRMs).
			Promoters were extracted 1000 base pairs (bp) upstream of the annotated Araport 11 \autocite{chengAraport11CompleteReannotation2017} TSS or until the nearest gene.
			5UTRs were extended downstream of the TSS to the closest coding region.
			Data points are positioned in the centre of each 100 bp window.
			Windows are offset by 50 bp.
			Shading represents 95 confidence intervals estimated using 10000 bootstraps.
			A: Median percentage of open chromatin peaks overlapping 100 bp windows. Open chromatin peaks derived from the intersect of root and shoot peaks derived from negative control (treated with NaOH) ATAC\hyp{}seq data by \textcite{potterCytokininModulatesContextdependent2018}.	
			B: Median percentage GC content in 100 bp windows. N = 95
			\label{fig:tfbs-coverage-sliding-window}
		}
	\end{center}
\end{figure}


Further analysis in this 400 bp region revealed a significant difference between promoter types (Kruskal\hyp{}Wallis~\textit{H} = 10.2,~\textit{P} \textless{} 0.01). Dunn's posthocs with Bonferroni correction showed that variable promoters had a significantly higher percentage bp covered (\SI{26.1}{\percent} ± 14.6) than constitutive promoters (\SI{20.0}{\percent} ± 13.7; \textit{P} \textless 0.01; \autoref{fig:tfbs-coverage-400bpprom}).

\begin{figure}[hbt!]
	\begin{center}
		\capstart
		\includegraphics[width=0.60\columnwidth]{bp_covered/400bpprom/Czechowski_TFBS_coverage_box}
		\caption{
			Percentage GC content of 100 constitutive (blue), 100 variable (orange) and 100 control (green) Arabidopsis \textit{cis}\hyp{}regulatory modules (CRMs) in a 400 bp region upstream of the ATG start codon.
			Box plots have box boundaries that represent 25th, 50th (median) and 75th percentiles; whiskers are drawn up to the largest or smallest observed point that falls within 1.5 times the interquartile range.
			\label{fig:tfbs-coverage-400bpprom}
		}
	\end{center}
\end{figure}

\subsection{TF diversity}
The hypothesis that constitutive genes have a higher diversity of TFs binding them than variable genes was tested.
There was no significant difference in TF (Kruskal\hyp{}Wallis~\textit{H} = 1.1,~\textit{P} \textgreater{} 0.05) or TF family diversity (Kruskal\hyp{}Wallis~\textit{H} = 0.3,~\textit{P} \textgreater{} 0.05) between promoter types \autoref{fig:tf-diversity-wholeprom}).

\begin{figure}[hbt!]
	\begin{center}
		\capstart
		\includegraphics[width=0.6\columnwidth]{TF_diversity/wholeprom/Czechowski_TF_diversity_box_subplots}
		\caption{
			Shannon diversity of individual TFs (A) and TF families (B) of 100 constitutive (blue), 100 variable (orange) and 100 control (green) Arabidopsis \textit{cis}\hyp{}regulatory modules (CRMs).
			Promoters were extracted 1000 bp upstream of the annotated Araport 11 \autocite{chengAraport11CompleteReannotation2017} TSS or until the nearest gene.
			5UTRs were extended downstream of the TSS to the closest coding region.  Box plots have box boundaries that represent 25th, 50th (median) and 75th percentiles; whiskers are drawn up to the largest or smallest observed point that falls within 1.5 times the interquartile range.			
			\label{fig:tf-diversity-wholeprom}
		}
	\end{center}
\end{figure}

A sliding window analysis revealed that in the constitutive open chromatin region 50-150 bp upstream of the ATG start codon (\autoref{fig:tf-diversity-sliding-window}A), variable promoters looked to have slightly higher Shannon diversity than constitutive promoters (\autoref{fig:tf-diversity-sliding-window}B). Median TF family Shannon diversity did not differ from 0 at any point along the CRM (\autoref{fig:tf-diversity-sliding-window}C).

 \begin{figure}[hbt!]
	\begin{center}
		\capstart
		\includegraphics[width=1\columnwidth]{TF_diversity/Czechowski_genetypenocontrol_TF_diversity_rw_median_sliding_window_combined}
		\caption{
			Sliding window analysis of 100 constitutive (blue) and 100 variable (orange) Arabidopsis \textit{cis}\hyp{}regulatory modules (CRMs).
			Promoters were extracted 1000 base pairs (bp) upstream of the annotated Araport 11 \autocite{chengAraport11CompleteReannotation2017} TSS or until the nearest gene.
			5UTRs were extended downstream of the TSS to the closest coding region.
			Data points are positioned in the centre of each 100 bp window.
			Windows are offset by 50 bp.
			Shading represents 95 confidence intervals estimated using 10000 bootstraps.
			A: Median percentage of open chromatin peaks overlapping 100 bp windows. Open chromatin peaks derived from the intersect of root and shoot peaks derived from negative control (treated with NaOH) ATAC\hyp{}seq data by \textcite{potterCytokininModulatesContextdependent2018}.	
			B: Median Shannon diversity of individual TFs in 100 bp windows.
			C: Median Shannon diversity of TF families in 100 bp windows.
			\label{fig:tf-diversity-sliding-window}
		}
	\end{center}
\end{figure}

Further analysis in the 400 bp region upstream of the start codon revealed a significant difference in individual TF diversity between promoter types (Kruskal\hyp{}Wallis~\textit{H} = 6.1,~\textit{P} \textless{} 0.05).
Dunn's posthocs with Bonferroni correction showed that variable promoters had a significantly higher Shannon diversity of individual TFs binding them (\SI{1.6}{\percent} ± 0.7) than constitutive promoters (\SI{1.3}{\percent} ± 0.8; \autoref{fig:tf-diversity-400bpprom}A).
There was no significant difference in the TF family Shannon diversity binding promoters between promoter types (Kruskal\hyp{}Wallis~\textit{H} = 3.3,~\textit{P} \textgreater{} 0.05; \autoref{fig:tf-diversity-400bpprom}B).


\begin{figure}[hbt!]
	\begin{center}
		\capstart
		\includegraphics[width=0.60\columnwidth]{TF_diversity/400bpprom/Czechowski_TF_diversity_box_subplots}
		\caption{
			Shannon diversity of individual TFs (A) and TF families (B) of 100 constitutive (blue), 100 variable (orange) and 100 control (green) Arabidopsis \textit{cis}\hyp{}regulatory modules (CRMs) in a 400 bp region upstream of the ATG start codon.
			Box plots have box boundaries that represent 25th, 50th (median) and 75th percentiles; whiskers are drawn up to the largest or smallest observed point that falls within 1.5 times the interquartile range.
			\label{fig:tf-diversity-400bpprom}
			*, \textit{P} \textless{} 0.05. ns, not significant.
		}
	\end{center}
\end{figure}

From \textasciitilde{}350 bp to \textasciitilde{}650 bp upstream of the start codon the median open chromatin in constitutive CRMs decreases to zero while variable CRMs had a median percentage open chromatin of 0.
To confirm whether this was a genuine difference or due to different 5'UTR lengths between constutitive and variable genes, the percentage of root\hyp{}shoot intersect open chromatin of windows was plotted centred around the Araport11 TSS (\autoref{fig:all-combined-sliding-window-araporttss}).
The median percentage open chromatin decreased upstream of the TSS for both constitutive and variable genes, and constutive genes had a higher percentage open chromatin -100 to + 500 bp around the TSS.%Add stats






%TALK ABOUT 5UTR lengths - test if constutive have longer 5'UTR lengths (check for expected  hypothesis first and see if useful comparison or not)
%\begin{figure}[!h]
%	\begin{center}
%		\capstart
%		\includegraphics[width=1\columnwidth]{slidingwindow_combined/Czechowski_genetypenocontrol_all_combined_rw_median_sliding_window}
%		\caption{
%			Sliding window analysis of 100 constitutive (blue) and 100 variable (orange) Arabidopsis \textit{cis}\hyp{}regulatory modules (CRMs).
%			Promoters were extracted 1000 base pairs (bp) upstream of the annotated Araport 11 \autocite{chengAraport11CompleteReannotation2017} TSS or until the nearest gene.
%			5UTRs were extended downstream of the TSS to the closest coding region.
%			Data points are positioned in the centre of each 100 bp window.
%			Windows are offset by 50 bp.
%			Shading represents 95 confidence intervals estimated using 10000 bootstraps.
%			A: Median percentage of open chromatin peaks overlapping 100 bp windows. Open chromatin peaks derived from the intersect of root and shoot peaks derived from negative control (treated with NaOH) ATAC\hyp{}seq data by \textcite{potterCytokininModulatesContextdependent2018}.
%			B: Position of Araport11 annotated transcription start site (TSS) upstream of the ATG start codon of the closest coding region.			
%			C: Median percentage GC content in 100 bp windows. N = 95
%			D: Median percentage bp covered by at least one transcription factor binding site.
%			E: Shannon diversity of individual transcription factors binding within each 100 bp window.
%			\label{fig:all-combined-sliding-window}
%		}
%	\end{center}
%\end{figure}

\begin{figure}[!h]
	\begin{center}
		\capstart
		\includegraphics[width=1\columnwidth]{slidingwindow_combined/Araport11_Czechowski_genetypenocontrol_median_openchromatin_sliding_window}
		\caption{
			Sliding window analysis of 100 constitutive (blue) and 100 variable (orange) Arabidopsis \textit{cis}\hyp{}regulatory modules (CRMs). Promoters were extracted 1000 bp upstream of the annotated Araport 11 \autocite{chengAraport11CompleteReannotation2017} TSS or until the nearest gene.
			5UTRs were extended downstream of the TSS to the closest coding region.
			TSS is represented by 0 on the x-axis.
			Data points are positioned in the centre of each 100 bp window.
			Windows are offset by 50 bp.
			Shading represents 95 confidence intervals estimated using 10000 bootstraps.
			A: Median percentage of open chromatin peaks overlapping 100 bp windows. Open chromatin peaks derived from the intersect of root and shoot peaks derived from negative control (treated with NaOH) ATAC\hyp{}seq data by \textcite{potterCytokininModulatesContextdependent2018}.		
			B: Median percentage GC content in 100 bp windows.
			C: Median percentage bp covered by at least one transcription factor binding site.
			D: Shannon diversity of individual transcription factors binding within each 100 bp window.
			\label{fig:all-combined-sliding-window-araporttss}
		}
	\end{center}
\end{figure}
\end{document}


