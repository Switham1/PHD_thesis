\documentclass[../main.tex]{subfiles}

\begin{document}

\chapter{Adding novel genetic variation to \textit{Cis}\hyp{}regulatory modules disrupts regulatory network}\label{chapter7}
\section{Introduction}\label{chapter7:introduction}
Most genome editing to date has focussed on mutating coding sequences.
For example, Gn1a regulates the number of reproductive organs in rice, and GS3 regulates grain size.
Mutations in \textit{GS3} and \textit{Gn1a} coding sequences lead to an increase in grain size and grain number \autocite{shenQTLEditingConfers2018}.
In the cucumber, the translation initiation factor eIF4E translates the RNA of several viruses.
Loss of function of this gene confers virus resistance \autocite{chandrasekaranDevelopmentBroadVirus2016}.
In tomato, LBD40 has a role in lateral organ development and jasmonate signalling.
Loss of function of this gene resulted in increased drought tolerance \autocite{liuCRISPRCas9Targeted2020}.
Engineering of \textit{cis}\hyp{}regulatory regions is an emerging strategy.
Compared to mutations in coding sequences that alter protein structure, \textit{cis}\hyp{}regulatory variants are often less pleitropic and cause subtle phenotypic chances by modifying the pattern, timing or strength of gene expression \autocite{wittkoppCisregulatoryElementsMolecular2012}.
The rice \textit{SWEET14} gene codes for a sucrose transporter, and is also susceptible to bacterial blight.
Disruption of the CDS confers blight resistance but plants have reduced growth as they lose function of the sugar transporter.
4 bp and 9 bp modifications to the \textit{SWEET14} promoter conferred blight resistance without significantly affecting plant growth \autocite{liHighefficiencyTALENbasedGene2012}.
The CLAVATA-WUSCHEL feedback signalling pathway controls meristem size \autocite{somssichCLAVATAWUSCHELSignalingShoot2016}.
WUS is expressed in the organising centre of the meristem where it increases the expression of the secreted peptide CLV3.
CLV3 binds to the receptor CLV1 which repressed \textit{WUS} expression.
In tomato, mutations to the \textit{CLV3} and \textit{WUS} promoters cuased increase locule number and fruit size due to stem cell overproliferation \autocite{rodriguez-lealEngineeringQuantitativeTrait2017}.
Recently, the promoter of the \textit{CLV3} orthologue \textit{CLE7} in maize was mutated \autocite{liuEnhancingGrainyieldrelatedTraits2021}.
Several mutations increased grain yield, while one mutation decreased grain yield.

Precise targeting of promoters leading to beneficial traits requires knowledge of which edges bind to each specific location in promoter.
Not all edges targeting promoters are known and it is currently not feasible to test every connection experimentally.
Therefore, adding variation across promoters using random mutagenesis is desirable.
To this aim, a CRISPR library approach to add variation to the promoters of four N-response master regulators in Arabidopsis was initiated.

\section{Aims}\label{chapter7:aims}
The aim of this chapter is to add novel genetic variation to promoter regions to disrupt edges.

\section{Results}\label{chapter7:results}
Results from chapter \ref{chapter3} showed that the region closest to the TSS or start codon is more important than regions further upstream, as architectural differences here explained the different expression patters of different gene categories.
Additionally, promoters containing more open chromatin were more actively expressed, and with motifs falling within open chromatin were more important.
Therefore, \textit{cis}\hyp{}regulatory module regions close to the start codon and overlapping open chromatin were chosen to increase the chance of disrupting important edges.
\end{document}