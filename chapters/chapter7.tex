\documentclass[../main.tex]{subfiles}

\begin{document}

\chapter{Adding novel genetic variation to \textit{Cis}\hyp{}regulatory modules disrupts regulatory network}\label{chapter7}
\section{Introduction}\label{chapter7:introduction}

Precise targeting of promoters leading to beneficial traits requires knowledge of which edges bind to each specific location in promoter.
Not all edges targeting promoters are known and it is currently not feasible to test every connection experimentally.
Therefore, adding variation across promoters using random mutagenesis is desirable.
To this aim, a CRISPR library approach to add variation to the promoters of four N-response master regulators in Arabidopsis was initiated.
%see Nicola notes 15/4/21 and add here
\section{Aims}\label{chapter7:aims}
The first aim of this chapter is to successfully add variation into the promoter regions of four N\hyp{}response master regulators using a CRISPR library approach.
The second aim is to study the effect of edge disruption in promoter regions on transcriptomics and phenotype, and to potentially explain any phenotypical changes by experimental validation of TFBS binding affinity changes.

\section{Results}\label{chapter7:results}
Results from chapter~\ref{chapter3} showed that the region closest to the TSS or start codon is more important than regions further upstream, as architectural differences here explained the different expression patters of different gene categories.
Additionally, promoters containing more open chromatin were more actively expressed, and with motifs falling within open chromatin were more important.
Therefore, \textit{cis}\hyp{}regulatory module regions close to the start codon and overlapping open chromatin were chosen to increase the chance of disrupting important edges.
The selected promoter regions of ARF9, ARF18, DREB26 and NLP7 were scanned for SpCas9 PAM sites, and 122 sgRNA cassettes were designed and constructed.
The guides were split into 96 pairs \textasciitilde{}100 bp apart, and 96 final Cas9 plasmids each containing a guide pair were constructed.
96 strains of Agrobacteria were transformed and were then mixed equally and used to transform \textasciitilde{}100 Col-0 Arabidopsis plants using Agrobacterium\hyp{}mediated floral dip.
Seeds were collected, germinated on kanamycin selective media and 200 T1 lines were transplanted to soil.
All 200 lines have been individually bagged ready for seed collection.

\section{Discussion}\label{chapter7:discussion}

\end{document}