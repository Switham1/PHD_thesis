\documentclass[../main.tex]{subfiles}

\begin{document}

\chapter{Rational design of minimal synthetic nitrogen-responsive promoters}\label{chapter5}
\section{Introduction}\label{chapter5:introduction}
My colleague Yaomin Cai recently tested the expression of minimal synthetic constitutive promoters (minsyns) in plant protoplasts using the dual luciferase assay \autocite{caiRationalDesignMinimal2020}.
Most of the TFs that bind to constitutive promoters were not themselves constitutively expressed.
Constitutive expression is the result of being able to utilise available groups of TFs in different cell types.
This supports an study, where deletion of different sections of constitutive promoters led to various different patterns of tissue-specificity \autocite{benfeyTissuespecificExpressionCaMV1990}.
Additionally, the use of different groups of proteins across tissue types might explain why transcripts from constitutive promoters tend to lack a conserved transcriptional start site relative to tissue specific promoters, as observed by %add Megraw lab paper here
\textcite*{caiRationalDesignMinimal2020} also found that the position of a proposed pioneer common \textit{cis}\hyp{}regulatory element (CRE) was important to the activity of the minsyns.
In absence of the pioneer CRE, transcription was activated via passive cooperativity.
When the order of TFBSs was changed, the expression did not significantly change.
This supported the passive cooperativity model of opening chromatin, ruling out TF-TF interactions between the TFs binding the minsyns.
It was also possible to predict the strength of minimal synthetic promtoers from their sequence.
Using a similar approach to \textcite*{caiRationalDesignMinimal2020}, I would like to design synthetic minimal nitrogen-responsive promoters for use in a synthetic genetic feedback controller (Chapter \ref{chapter6}: Adding synthetic genetic feedback control).
\section{Aims}\label{chapter5:aims}
The aim is to design, build and quantitatively test the expression of several synthetic minimal N\hyp{}responsive promoters using a dual luciferase ratiometric assay in Arabidopsis protoplasts.
Ideally several promoters of different strengths responding to different N\hyp{}responsive TFs will be generated.
Some of these synthetic N\hyp{}responsive promoters will then be used to control synthetic TFs in a genetic feedback controller in chapter~\ref{chapter6}.
\section{Results}\label{chapter5:results}

\end{document}