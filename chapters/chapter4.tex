\documentclass[../main.tex]{subfiles}

\begin{document}

\chapter{Validation of TF-promoter interactions}\label{chapter4}
\section{Introduction}\label{chapter4:introduction}
Recently, systems biology approaches have identified a nitrogen response subnetwork \autocite{gaudinierTranscriptionalRegulationNitrogenassociated2018}.
This is composed of many network motifs such as feed forward loops (FFLs).
%Add figure of network and subnetwork
We would like to validate the subnetwork to characterise the role of network components and their connections (edges) so that they can be rationally and predictably engineered.
Yeast one-hybrid and Dap-seq experimental techniques suggest hypotheses for edges but they are not definitive as they do not determine whether edges activate or repress their target genes.
We are using an in-house method called TRAMP to quantitatively determine TF-DNA binding affinity (lab work done by Yaomin Cai).
Tufan Oz and I set up the TARGET~\autocite{bargmannTARGETTransientTransformation2013} method in the lab to determine direct and indirect targets of nodes and whether they activate or repress each target. Since setting up the protocol, Tufan carried out all of the work.
Using TRAMP and TARGET assays we would like to validate edges in the nitrogen response network.
\section{Aims}\label{chapter4:aims}
To validate edges in the N-response GRN and test whether they activate or repress their targets using TARGET~\autocite{bargmannTARGETTransientTransformation2013}.
To validate the binding affinity of N-reponse TFs to their proposed TFBS in promoters of their targets using the TRAMP assay.
\section{Results}\label{chapter4:results}

\end{document}