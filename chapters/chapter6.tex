\documentclass[../main.tex]{subfiles}

\begin{document}

\chapter{Adding synthetic genetic feedback control}\label{chapter6}
\section{Introduction}\label{chapter6:introduction}
Network motifs are patterns of gene regulation in GRNs which occur more often than by chance \autocite{miloNetworkMotifsSimple2002}.
These include positive and negative autoregulation, cascades, feedforward loops (FFLs) and feedback loops \autocite{shovalSnapShotNetworkMotifs2010}.
FFLs consist of a regulator, X, which regulates Y and Z, where Y also regulates Z.
There are 8 possible combinations of FFL \autocite{shovalSnapShotNetworkMotifs2010}.
In \textit{E. coli} \autocite{shen-orrNetworkMotifsTranscriptional2002} and \textit{S. cerevisiae} \autocite{leeTranscriptionalRegulatoryNetworks2002}, human \autocite{boyerCoreTranscriptionalRegulatory2005} and Arabidopsis \autocite{chenArchitectureGeneRegulatory2018} GRNs, the feedforward loop is the most common motif.
Addition of a negative-feedback loop to a FFL has been predicted to improve the stability and robustness of the expression dynamics, facilitating the return to the original steady state before addition of the stimulus \autocite{reevesEngineeringPrinciplesCombining2019}.
This return to the original steady state is called perfect adaptation.
Gene expression dynamics are influenced by network motifs, for example, negative\hyp{}feedback loops can generate oscillations while \hyp{}feedback loops can promote bistability \autocite{shovalSnapShotNetworkMotifs2010}.
\section{Aims}\label{chapter6:aims}
I would like to add a positive\hyp{}feedback loop into the N-response GRN to promote bistability and robustness in the presence of noise. I would also like to add a negative\hyp{}feedback loop to generate oscillations in gene expression.
By using synthetic N-responsive promoters of different stengths responding to different TFs, I hope to be able to tune the expression dynamics to increase or decrease the robustness of the GRN in responding to nitrate.
\section{Results}\label{chapter6:results}

\end{document}