\documentclass[../main.tex]{subfiles}

\begin{document}
\chapter{Progress since May 2020}\label{progress}

\textbf{Adding novel genetic variation to promoter regions to disrupt edges}
Four nitrogen master regulators were identified. \textasciitilde{}500 bp open chromatin regions in each promoter were identified. These regions were scanned for SpCas9 PAMs sites. 122 guides were designed ensuring predicted on-target efficiency of \textgreater{} \SI{40}{\percent}, with \textasciitilde{}30 guides per promoter. A custom Python script allocated each guide into a pair 90-110 bp apart with the aim of increasing the variety of mutations, with a chance to cause a large deletion between the two PAM sites in a pair. 122 sgRNA cassettes were constructed and used to build 96 final Cas9 constructs each containing a guide pair using Golden Gate assembly. 96 strains of Agrobacteria were transformed with the plasmids. These strains were combined equally and then sprayed onto ~100 Col-0 Arabidopsis thaliana plants using the Agrobacterium-mediated floral dip method. Seeds were collected, grown on selectable kanamycin media and 200 T1 lines were transferred to soil. These plants have now been bagged ready for seed collection. In the future 500 T2 lines will be grown and genotyped using a CRISPR sequencing approach. Plants containing a range of variation in the promoter regions of interest will be grown to T3 level for phenotyping and expression analysis using RNA-seq.

\textbf{Adding network motifs to the N subnetwork}
Another aim is to add genetic feedback into the N subnetwork to influence expression dynamics, promoting bistability, efficient switching behaviour, robustness in the presence of noise, and tunability. NLP7 is regulated by ARF18 through ANAC032. To add genetic feedback from NLP7 to ARF18, a synthetic promoter responding only to NLP7 controlling a synthetic TF which binds to and regulates ARF18 is desired. To design synthetic promoters, more information is needed about promoter architecture. To this aim, a comprehensive bioinformatics analysis was undertaken to elucidate the complex promoter architectural differences between constitutive and variable genes, and tissue-specific and non-specific genes.

\textbf{Understanding promoter architecture to allow engineering of synthetic promoters}
I added tissue-specific and non-specific promoter categories using Tau tissue-specificity ranking to my promoter architecture analysis because the coefficient of variation of expression ranking excluded tissue-specific genes. I scanned promoters using a sliding window approach to identify regions of importance for explaining differences in expression between the gene categories. A 400 bp region upstream from the ATG start codon was chosen and statistically analysed. I found a significant difference in TFBS percent coverage between groups in the 400 bp promoter region. In the same region GC content was higher in constitutive promoters than variable, and higher in non-specific than tissue-specific promoters. TATA boxes were enriched in variable promoters compared to the background, and negatively enriched in constitutive promoters compared to the background.
For engineering synthetic promoters the region closest to TSS/start codon is more important as differences here explain the different expression patterns of the different gene categories. TATA boxes should be included in responsive promoters. Promoters with more open chromatin are more actively expressed, and motifs falling within open chromatin are more important. This was useful to know for the CRISPR library approach, where only variation was added to promoter regions falling within open chromatin to ensure important regions were disrupted.

\textbf{Adding synthetic genetic feedback to the N subnetwork}
Several different synthetic N-responsive promoters were designed and assembled. These respond to either NLP7, TGA1, bZIP3 (activators) or HHO2 (repressor). Promoters with combinations of NLP7 and TGA1 or NLP7 and bZIP3 were also assembled. The response of these promoters to the TFs that bind them was tested in Arabidopsis leaf protoplasts using a dual luciferase assay. These experiments will be repeated using the same batch of plants as variation was high between batches. In the near future the synthetic TFs will be assembled and tested in protoplasts using qPCR, to make sure they activate or repress ARF18 as designed. Once all parts have been tested, the synthetic genetic feedback circuit will be constructed, with the synthetic promoter controlling the synthetic TF which then activates or represses ARF18. Transcriptomics of important N master regulators will be tested transiently in protoplasts using qPCR. The circuit will also be tested in stable lines where expression and phenotype analysis will be conducted.

\textbf{Timeline}